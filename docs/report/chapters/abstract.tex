\begin{abstract}
    \addcontentsline{toc}{chapter}{General Introduction}
    \par With the recent progress in technology and Artificial Intelligence in particular, businesses became increasingly interested in leveraging the domains of application of these advancements in order to optimize their internal processes and the efficiency and productivity of their employees. The emergence of Generative AI chatbots (e.g., ChatGPT, Google Gemini, etc...) has opened up exciting new perspectives in this domain; It came up with an alternative approach for searching large volumes of information, delivering significantly more insightful results in less time and effort than traditional methods.\medskip
    \par This report presents the design and implementation of an Entreprise Knowledge Base searching tool to facilitate the quick and precise retrieval of custom and proprietary information by employees, thereby enhancing knowledge-intensive workflows.
    This solution was conceived in fulfillment of an end-of-studies internship project at "Standard Sharing Software", specifically tailored to the company's requirements. It provides users with functionalities to augment and organize the knowledge base, enabling them to retrieve the most relevant contents thereof when needed.\smallskip
    \par To fulfill the purposes of this solution, various technologies have been explored and implemented: A Vector Store for persisting documents into the knowledge base, embeddings and similarity search algorithms for retrieving the most relevant passages corresponding to user queries, Large Language Models (LLMs) to synthesize these retrieved passages, condensing them into concise and understandable responses and Prompt Engineering for the optimal utilization of LLMs' capabilities, among other techniques. The synergistic orchestration of these diverse tools constitutes the Retrieval-augmented Generation (RAG) framework.\medskip
    \par This report first details the context and the compelling need for such a solution within the company. It then presents the conception phase of the project, highlighting the functional and non-functional requirements along with the general use case diagram and planning out tasks and deadlines. It subsequently introduces the various technical terms employed throughout the project, elucidating the rationale behind the selection and application of these technologies. Finally, it concludes by detailing the implementation process and the achieved results, while also acknowledging potential limitations and avenues for further improvement.\bigskip
    \par\textit{\textbf{Keywords:} Retrieval-augmented Generation (RAG), Large Language Models (LLMs), Generative AI, Vector Store, Embeddings, Similarity Search, Prompt Engineering}
\end{abstract}