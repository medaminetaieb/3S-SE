\chapter*{Glossary}
\markboth{Glossary}{Glossary}
\addcontentsline{toc}{chapter}{Glossary}
\begin{itemize}
    \item \textbf{3S} : Standard Sharing Software (The hosting company)
    \item \textbf{RAG} : Retrieval-Augmented Generation
    \item \textbf{LLM} : Large Language Model
    \item \textbf{EKB} : Entreprise Knowledge Base
    \item \textbf{AI} : Artificial Intelligence
    \item \textbf{GPT} : Generative Pre-trained Transformer
    \item \textbf{NLP} : Natural Language Processing
    \item \textbf{RL} : Reinforcement Learning
    \item \textbf{RLHF} : Reinforcement Learning from Human Feedback
    \item \textbf{SBERT} : Sentence Bidirectional Encoder Representations from Transformers (or Sentence Transformers for short)
    \item \textbf{HF} : Hugging Face
    \item \textbf{MTEB} : Massive Text Embedding Benchmark
    \item \textbf{ML} : Machine Learning
    \item \textbf{DL} : Deep Learning
    \item \textbf{URL} : Uniform Resource Locator
    \item \textbf{API} : Application Programming Interface
    \item \textbf{OS} : Operating System
    \item \textbf{CPU} : Central Processing Unit
    \item \textbf{GPU} : Graphics Processing Unit
    \item \textbf{RAM} : Random-access memory
    \item \textbf{DB} : Database
    \item \textbf{CUDA} : Compute Unified Device Architecture
    \item \textbf{VM} : Virtual Machine
    \item \textbf{WSL} : Windows Subsystem for Linux
    \item \textbf{JS} : JavaScript
    \item \textbf{HTML} : Hypertext Markup Language
    \item \textbf{a.k.a.} : also known as
    \item \textbf{e.g.} : for example
    \item \textbf{i.e.} : that is
\end{itemize}